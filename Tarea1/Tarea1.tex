\documentclass{article}
\usepackage[utf8]{inputenc}
\usepackage{amsmath,amsfonts,amssymb}
\usepackage{geometry}
\usepackage{color}
\usepackage[spanish]{babel}
\geometry{letterpaper, margin=1in}
\usepackage{listings}
\lstset{
  language=Python,
  basicstyle=\small\ttfamily,
  keywordstyle=\color{blue},
  commentstyle=\color{green},
  stringstyle=\color{red},
  breaklines=true,
  showstringspaces=false,
  numbers=left,
  numberstyle=\tiny,
  frame=lines
}
\usepackage{setspace} \doublespacing

\title{Tarea 1}
\author{Fernando Cruz Pineda, Edson Rafael Flores}
\date{\today}
\begin{document}
\maketitle

\begin{enumerate}
\item Muestra una máquina de Turing total con alfabetos de entrada y cinta $\Sigma = \{1,\#\}$ y  $\Gamma=\{1,\#,x,\sqcup \}$, respectivamente, que reconozca el lenguaje:

  $A = \{\alpha \# \beta \gamma | \alpha,\beta,\gamma \in \{1\}^* \land |\alpha| + |\beta| = |\gamma| \}$

  Descrito en palabras, el lenguaje A tiene todas las ternas de enteros en notación unaria tales
que la suma de los dos primeros números es igual al tercer número.
Primero da una descripción de alto nivel de tu máquina (como referencia, toma las descripciones de alto nivel de los ejemplos 3.7 a 3.12 en el libro de Michel Sipser), después da la descripción formal, y finalmente da una explicación de porque la máquina es total y reconoc el lenguaje A.

\item Sea $\Sigma$ un alfabeto finito. Demuestra que para todo lenguaje reconocible $A \subseteq \Sigma ^*$ existe una máquina de Turing M tal que L(M) = A y siempre que M rechaza una cadena de entrada $w \in \Sigma ^*$, lo hace entrando en un loop infinito.

  Dem.

  Sea $A \subseteq \Sigma^*$ un lenguaje reconocible cualquiera, esto quiere decir que existe una MT $N = (Q, \Sigma, \Gamma, \delta, q_0, q_{acepta}, q_{rechaza})$, tal que $L(N) = A$.

  Proponemos la siguiente máquina de Turing $M = (Q', \Sigma, \Gamma, \delta_M, q_0, q_{acepta})$ tal que

  $Q' = Q - \{q_{rechaza}\} \cup \{p\}$

  
  $\delta(q,x)_M = $
  \begin{cases}
    (p,a,L) \text{ si } \delta(q,x) = (q_{rechaza},a,L) \\
    (p,a,R) \text{ si } \delta(q,x) = (q_{rechaza},a,R) \\
    (p,x,R) \text{ si } q = p \\
    \delta(q,x)  \text{ en otro caso}
  \end{cases} para cualquier $x,a \in \Sigma$ y $q \in Q'$

  P.d $L(M)=A$

  Procedemos por doble contención
  
  P.d $A \subseteq L(M)$

  Sea $w \in A$ una cadena cualquiera, como A es reconocible, significa que w es aceptada por N. Esto implica que N llega al estado $q_{acepta}$ tras procesar esta cadena. Por la construcción de M, las transiciones de M que conducen a la aceptación son idénticas a las de N. Por lo tanto, si N acepta w, M también lo hará, así $w \in L(M)$

  P.d $L(M) \subseteq A$

  Sea $w \in L(M)$, esto quiere decir que M acepta a w. Por la construcción de M, M acepta si y sólo si N acepta, por lo tanto $w \in L(N) $ pero $L(N)=A$, así $w \in A$.

  Por lo que podemos concluir que $L(M)=A$.

  P.d Cuando M rechaza lo hace entrando a un loop infinito

  Sea w una cadena que es rechazada por N, esto sucede unicamente si la computación de w en N termina en el estado $q_{rechaza}$, ahora, según nuestra construcción de M, cualquier transición en N que hubiera llevado a $q_{rechaza}$ lleva al estado p en M. Una vez en el estado p, según la definición de $\delta_M$, no importa lo que se lea M se va a quedar en p y se va a mover a la derecha sin parar. Dado que p no es un estado de aceptación ni de rechazo, la máquina nunca se detiene. Es decir, se queda en un loop infinito.

  Así podemos concluir que para cualquier lenguaje A reconocible existe una MT M tal que $L(M)=A$ y M rechaza entrando a un loop infinito.
  
\item Sea $\Sigma$ un alfabeto finito. Demuestra que si $A \subseteq \Sigma ^*$ es un lenguaje regular, entonces A es decidible.

  

  
  \end{enumerate}
\end{document}
